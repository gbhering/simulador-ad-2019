\documentclass{article}
\usepackage[utf8]{inputenc}
\usepackage[brazilian]{babel}
\usepackage[utf8]{inputenc}
\usepackage[T1]{fontenc}
\usepackage{amssymb}
\usepackage{url}
\usepackage{listings}
\usepackage{xcolor} 

\newcommand*\barra[2][]{\overline{ #1_{#2} }}

\lstset{
    frame=tb, 
    tabsize=2,
    showstringspaces=false,
    numbers=left, 
    commentstyle=\color{gray}, 
    keywordstyle=\color{blue}, 
    stringstyle=\color{red} 
}

\title{Trabalho de Simulação de MAB-515}
\author{ Gabriel Bhering Dominoni }
\date{12 de Julho de 2019}

\begin{document}

\maketitle
\thispagestyle{empty}

\pagebreak

\tableofcontents

\pagebreak

\section{Introdução}
O programa foi implementado em C++, utilizando de todos os recursos da linguagem para máxima eficiência. A constantes do programa - lambda, disciplina e kmin - são definidas em código pré-compilação para que o compilador possa otimiza-las. \\

Os eventos são armazenados numa fila de prioridade que ordena eventos cronologicamente. A função principal insere a primeira chegada, e roda a simulação R vezes, resultando em vetores com as estatisticas de cada rodada. As variancias e médias entre rodadas são então calculadas, junto com seus intervalos de confiança. \\

Há apenas dois tipos de evento nesta simulação: chegads e partidas. Em ambos os tipos de evento, estatisticas referentes ao número de pessoas em espera e tempo de espera são coletadas. Chegadas aumentam o número de pessoas na fila, e o número de pessoas em espera, além de agendar uma partida caso não haja outros fregueses na fila. Partidas naturamente diminuem os contadores de fila e espera além de agendar a próxima partida caso haja fregueses em espera. \\

\section{Testes de Correção}
A implementação 

\section{Estimativa da fase transiente}

\section{Resultados}
\subsection{Resultados Analíticos}
\begin{center}
\begin{tabular}{||c c c c c c||} 
 \hline
  $\lambda$ & $\barra{W}$ & $Var(W_{FCFS})$ & $Var(W_{FCFS})$ & $\barra{N_q}$ & $Var(N_{q})$ \\ [0.5ex] 
 \hline
0.2 & 0.25 & 0.5625 & 0.71875 & 0.05 & 0.0725 \\
0.4 & 0.66... & 1.77... & 3.259259... & 0.266... & 0.5511... \\
0.6 & 1.5 & 5.25 & 16.5 & 0.9 & 2.79 \\
0.8 & 4 & 24 & 184 & 3.2 & 18.56 \\
0.9 & 9 & 99 & 1719 & 8.1 & 88.29 \\
 \hline
\end{tabular}
\end{center}

\subsubsection{Fórmulas Analíticas}
\begin{center}
\begin{tabular}{||c c c||} 
 \hline
  & Média & Variância  \\ [0.5ex] 
 \hline
 W_{FCFS} & $\frac{\lambda}{1-\lambda}$ & $\frac{2\lambda-\lambda^2}{(1-\lambda)^2}$  \\ [2ex] 
 W_{FCFS} & $\frac{\lambda}{1-\lambda}$ & $\frac{2\lambda-\lambda^2+\lambda^3}{(1-\lambda)^3}$  \\ [2ex] 
 N_q & $\frac{\lambda^2}{1-\lambda}$ & $\frac{\lambda^2+\lambda^3-\lambda^4}{(1-\lambda)^2}$  \\ [2ex] 
 \hline
\end{tabular}
\end{center}

Conheçemos $N_q(z)$ (p. 91 apostila), conheçemos $N(z)$ e seus dois primeiros momentos (p. 90 apostila) e conheçemos $X^*$ (serviço exponencial com taxa 1). 

\[ N_q(z) = \frac{ N(z) - ( 1 - \lambda ) }{ z } \therefore N_q' (z) = \frac{N'(z)}{z} + \frac{N'(z)}{-z^2}\]

Substituindo $z=1$

\[ N_q' (1) = E[N] + N(z) - (1-\lambda)\]


\section{Conclusões}

\section{Anexo}

O código se encontra em: \url{github.com/gbhering/simulador-ad-2019}. Aqui se encontram descritas as rotinas mais relevantes do programa.

\begin{lstlisting}[language=C++, caption={Loop principal do programa}]
nova_chegada(); // primeira chegada

int lim = k;
while( eventos != tail && --lim) {
	auto e = pop();
	if ( e->tipo == chegada )
		nova_chegada();
	else if ( e->tipo == fregues )
		serve();
}
\end{lstlisting}

\end{document}


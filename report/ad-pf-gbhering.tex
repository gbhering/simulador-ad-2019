\documentclass{article}
\usepackage[utf8]{inputenc}
\usepackage[brazilian]{babel}
\usepackage[utf8]{inputenc}
\usepackage[T1]{fontenc}
\usepackage{url}
\usepackage{listings}
\usepackage{xcolor} 

\newcommand*\barra[2][]{\overline{ #1_{#2} }}

\lstset{
    frame=tb, 
    tabsize=2,
    showstringspaces=false,
    numbers=left, 
    commentstyle=\color{gray}, 
    keywordstyle=\color{blue}, 
    stringstyle=\color{red} 
}

\title{Trabalho de Simulação de MAB-515}
\author{ Gabriel Bhering Dominoni }
\date{8 de Julho de 2019}

\begin{document}

\maketitle
\thispagestyle{empty}

\pagebreak

\tableofcontents

\pagebreak

\section{Introdução}
O programa foi implementado em C++, utilizando de todos os recursos da linguagem para máxima eficiência. A constantes do programa, as taxas, disciplina, kmin e verbose, são definidas em código pré-compilação para dar a oportunidade do compilador otimiza-las. 

Como tradicionalmente, existe uma lista de eventos, e um laço que a processa. A lista de eventos está armazenada numa fila ordenada própria da linguagem. A ordenação muda de acordo com a disciplina escolhida. Especificamente, existem apenas dois tipos de evento - chegadas e partidas - e se a fila for FCFS, a fila é executada na ordem normal, se for LCFS, as partidas são executadas em ordem inversa.

Os eventos são estruturas básicas, contendo o tempo de entrada na fila, tipo e duração. Os tempos são contados em variáveis globais.

\section{Testes de Correção}
A implementação 

\section{Estimativa da fase transiente}

\section{Resultados}

\subsection{Deduções}

\subsubsection{Modelagem Geral}
\subsubsection{Tempo de Espera}
\subsubsection{Número de Pessoas em Espera}

\section{Conclusões}

\section{Anexo}

O código se encontra em: \url{github.com/gbhering/simulador-ad-2019}. Aqui se encontram descritas as rotinas mais relevantes do programa.

\begin{lstlisting}[language=C++, caption={Loop principal do programa}]
nova_chegada(); // primeira chegada

int lim = k;
while( eventos != tail && --lim) {
	auto e = pop();
	if ( e->tipo == chegada )
		nova_chegada();
	else if ( e->tipo == fregues )
		serve();
}
\end{lstlisting}

\end{document}

